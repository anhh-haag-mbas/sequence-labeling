\begin{abstract}
Sequence labelling is a fundamental task in natural language processing and
there exists several renowned machine learning models that achieve
state-of-the-art results on such tasks. In this thesis, we examine two such
models, the bidirectional LSTM and the bidirectional LSTM with an added
Conditional Random Field layer, and see how the perform on a number of different
languages. We limit the scope to include the part-of-speech tagging task and the
named entity recognition task, and we implement and review the models in three
different machine learning frameworks, DyNet, Pytorch and TensorFlow. We define
several configurations and perform a total of 3780 experiments, and we give an
introductory description of the theoretical and empirical background of the
models used and the parameters used in our configurations. Our results show,
that it is possible to achieve strong performance by applying CRF to the
standard bidirectional LSTM, but also that this advantage diminishes if 2
bidirectional LSTMs are combined. We also observe, that batch sizes have an
impact on both the accuracies of the models and the time it takes for the models
to converge. Furthermore, we find that the pretrained Polyglot word embeddings
is ill-suited for certain data sources for Japanese part-of-speech data.
Finally, we give suggestions for improvements based on our findings, and we give
individual reviews of the frameworks used for implemenations. 
\end{abstract}
