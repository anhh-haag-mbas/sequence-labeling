\begin{abstract}

Sequence labeling is a fundamental task in natural language processing and there
exists several renowned machine learning models that achieve state-of-the-art
results on such tasks. In this thesis, we examine two such models, a
bidirectional LSTM model with and without a Conditional Random Field, and see
how they perform on a number of different languages. We limit the scope to
include the part-of-speech tagging task and the named entity recognition task,
and we implement and review the models in three different machine learning
frameworks, DyNet, PyTorch and TensorFlow. We define several configurations and
perform a total of 3780 experiments, and we give an introductory description of
the theoretical and empirical background of the models and the parameters
used in our configurations. Our results show, that it is possible to achieve
strong performance by applying CRF to the standard bidirectional LSTM, but also
that this advantage diminishes if 2 bidirectional LSTMs are combined. We also
observe, that batch sizes have an impact on both the accuracies of the models
and the time it takes for the models to converge. Furthermore, we find that the
pre-trained Polyglot word embeddings are ill-suited for certain data sources for
Japanese part-of-speech data. Finally, we give suggestions for improvements
based on our findings, and we give individual reviews of the frameworks used for
implementations.

\end{abstract}
