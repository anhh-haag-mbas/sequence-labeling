
\section{Discussion}

\subsection{Possible sources of Discrepencies}

Different ways of splitting japanese (and others?) data, CoNNL-U seem to split
based on some ``smart'' method, whereas our BIO data is just single character
each. The polyglot embedding for japanese though might also be single character.

Differences in implementations of batching between framework. Autobatching
Mean vs sum of losses.

Takes different number of epochs to converge, some actually gets 50 epochs.

Differences in implementation of CRF.

The issue in PyTorch where not every sentence is used doesn't affect the
training since 1, 8, and 32 are whole dividers of the 4000 training sentences.

Different way to count number of epochs run

Without CRF we see that 2-layer Bi-LSTM performs better. With CRF the difference is negligible.
Especially as the batch size increases.

As expected, for a single epoch and 32 batch size, the models perform worse, 
but we see that dynet performs significantly worse than pytorch. This may be 
because of the 2-Layer Bi-LSTM in DyNet which requires more training before 
performing reasonably. It can also be explained by the issue where pytorch
doesn't pick the best model but rather the latest one.

Batch size have less influence on the performance given more training.
With higher batch size there are fewer updates (backpropogations) per epcoh,
this may also be interesting to research.

DyNet doesn't train as much when adding a CRF layer, possibly because of a bad
implementation or integration. Tensorflow similarly changes behaviour when using
CRF, but it can't stop training.

\subsection{Frameworks evaluation}
\label{subsec:frameworks}

Pros and cons of each framework.

\subsubsection{Dynet}

The performance gain when running on batches may be because of fewer
backpropogations rather than any form of parallelization.

DyNet performs significantly slower when run with a CRF layer, this may suggest
that the implementation used isn't optimized. 


% nvprof for profilling

\textbf{The learning proccess}

My general approach to learning dynet was to ease myself into it, following
simple tutorials like an XOR example, and building from there. With an
understanding of how a simple network works I could try to create my own on a
toy problem and see if I really understood how stuff worked. One of the nice
things about Dynet is how close it feels to creating computational graphs.
Creating parameters and linking them together by applying mathematical
operations on them, made it easy to reason about the behavior of the model.
After the small examples it was simple to learn how to work an LSTM layer into
the model. LSTM in dynet hides a lot of the complexity associated with the
computations, but the interface was still intuitive to use. Get the initial
state, compute the next state based on input, get the output for the next layer
and save the state for the next part of the sequence. Working out how to add a
CRF layer on top was mostly a matter of understanding CRF rather than Dynet.
Figuring out that what we needed for CRF was simply a transformation matrix,
made it obvious that this matrix should just be added as a a parameter similar
to other parts of a model. From there I could take inspiration from a complete
model implemented in dynet and reuse their code for the CRF.


\subsection{TensorFlow}

\subsubsection{Availability of information}

It is incredibly easy to find guides/blog posts/papers on how to use the Keras
interface for TF. It is furthermore incredibly easy to find code examples and
code for newer more experimental designs, such as CRF/LSTM, implementations for TF/Keras.

Finding guides/help for TF, and NOT Keras, is quite difficult.

\subsubsection{High-level nature of Keras}

Because of the high level nature of Keras, a lot of implementation details are hidden. This is fine if you are already perfectly familiar with the theory, but makes it harder to understand what the framework is actually doing if you are only somewhat familiar with the theory behind the different things  (LSTM, CRF, MLP, etc.).

Contrast this with other frameworks or just with using TF directly, where everything isn't conviently wrapped in high level "layers". Using these you are forced to understand what is actually going on, at least more than with Keras, just to get any result.

Coupling the high level nature of Keras with the abundance of guides, means that you could actually get started with working code within your given area, without understanding any of the code or theory.

That is undesirable.

On the other hand if you are intimatley familiar with the theory working with TF must be a breeze, as it abstracts all of the technical details away.

Working with non-standard "layers" is also quite easy, but does require a deeper understanding of how TF works.

\subsubsection{Workflow}

I have worked a lot by finding examples on various blog posts and then copy pasting them into a jupyter notebook.

I have then experimented with tweaking the examples and writing them together to gain a deeper understanding of TF works. I would often come across features of TF that I did not know existed, which would then prompt me to read the TF docs on that feature.

This allowed me to quite quickly get working results, but it limited just how deeply I got to understand the theory. That is because I relied on existing implementations of bi-lstm/crf, and the way I got started was by looking at how other people had already tackled the problems.

Had I developed the lstm and crf by hand, I would have been forced to gain a much much deeper understanding of how they worked and of how TF worked. This would however have required a much greater time dedication.

It was thus a weighting between time and deep understanding of TF/Keras/CRF/LSTM.


\subsection{Suggestions for further research}

Someone should really look into that urdu language. Also, whats up the the arabs
and the japanese??

\subsection{Improvements}

The models created in this project were kept simple to decrease unnecessary
complexity, as such a lot of changes can be made to improve their performance.
Here we present just a few of the changes that could be made that we are aware
of.

\subsubsection{Character embedding}

Similar to the way that pretrained word embeddings like Polyglot can be used to
significantly improve performance, including a character embedding, pretrained
or not, will also help the model make better predictions \ref{misconceptions
yang zhang}.

Character embeddings, like word embeddings, work by mapping a character to a
vector representation. But character embeddings can contain a mapping from every
character to a vector representation since there are generally not a lot of
different characters in a language. Even languages like Chinese with thousands
of different characters is not an issue. This gives the model better predictive
power on words which are not part of the word embedding.

There are a lot of different ways to include a character embedding into a model.
The general approach is to map each character in a word to its own vector
representation and use an additional layer to read the sequence of character
representations and return a single vector. This is similar to the way we use an
LSTM to read the sequence of words in a sentence, but instead we only use a
single value. The most commonly used way to read the sequence of characters are
LSTM or CNN layers. They perform very similarly, but CNN are a lot faster to
train \ref{misconceptions yang zhang}.

A character embedding can contain a mapping from not just a single character,
but combinations of two or more characters or so call n-grams. Since a lot of
languages use a relative small alphabet, a three character embedding could be
less than a million different values, which while big is still manageable.

The benefit of using a character embedding is that a model will always be able
to learn a unique representation for each word, whereas a word embedding can
only learn as many representations as the number of words it knows. An
additional benefit is that many words contain character level information which
a word embedding may not learn. As an example, words which are capitalized are
more likely to be named entities, than words which are uncapitalized, and words
which ends in ``ing'' are likely verbs.

\subsubsection{Stacked Bi-LSTM}

As has been mentioned before, the DyNet implementation uses a 2 layer Bi-LSTM.
This is sometimes refered to as a stacked Bi-LSTM \ref{Neural Network Methods
for Natural Language Proccessing ch. 14.5}. While not clear how exactly this
gives a better model accuracy, it has been shown empirically to help
\ref{Papers using 2 layer}.

An explanation could simply be that adding additional Bi-LSTM layers increases
the complexity of the model which allows it to represent more complicated
functions. Similar to the way adding additional linear layers to a model can
improve a model. While stacked linear layers is in many cases equivalent to
increasing the dimensionality of the weights and biases \ref{Deep learning 199,
Barron 1993} this may or may not be the case for Bi-LSTMs.

\subsubsection{DyNet manual batching}

For DyNet, we used autobatching to implement mini-batches. This is an algorithm
which based on the computational graph build, can create batches for optimized
training time. This shouldn't have any impact on the accuracy of the model, but
it improves the speed compared to running without. It is however still slower
than creating the batches manually and as such the DyNet implementation could be
improved in this sense if reducing the training time is important.


\subsection{Other stuff}

\begin{itemize}
    \item Handling, analyzing and presenting large amounts of data was
        something, that we had no prior experience in. This was evident in the
        final phase of our project, where we struggled with the tools and
        techniques to properly work with the data we had created.
    \item Choosing to implement the same models in three different frameworks
        had more of an educational purpose than a scientific one. Due to
        complications in understanding the inner work of each of our one
        frameworks, the task of synchronizing the implementations became
        disproportionally difficult. Our results are therefore burdened by the
        uncertainty of whether our programs actually the work, as they are
        intended to.
    \item It seems the Tensorflow implementation does not handle increasing
        batch sizes well. This is seen in Figure~\ref{chart:ner-acc-epo} and in
        Figure~\ref{chart:pos-acc-epo}. When training with early stopping using
        a patience of 3, Tensorflow looses accuracy for arabic, danish, hindu
        and urdu as the batch size increase from 1 to 8 and from 8 to 32. And
        except for japanese, a batch size of 1 yields the best accuracy on all
        other languages 
\end{itemize}

\pagebreak
