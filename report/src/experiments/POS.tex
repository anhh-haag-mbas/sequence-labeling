
\subsection{Part-of-speech tagging}

This section will go into the details of the Part-of-speech task, what the
specifics are, considerations in regards to the data used, any issues we came
across, and the results of our experiments.

\subsubsection{Task definition}

The Part-of-Speech task is a classification task where the objective is to label
each word in a sentence with it's corresponding part-of-speech label such as
noun, verb, adjective, etc. There are around 17 different labels, but some
languages don't use all of them. The same words can have different labels in
different context, so the classifier should be able to figure out which label is
a better fit on a sentence by sentence basis. Simply remembering that word $X$
has label $Y$ wouldn't be able to generalize very well.

The performance of a classifier for the POS task is the simple accuracy of the
predictions. Since there is no label which is a lot more common than others,
guessing randomly would very rarely give a better accuracy than 10\%.

\subsubsection{Data}

For this task we used datasets from~\ref{}{UniversalDependencies.org} which has
a broad selection of languages with multiple datasets (called treebanks) in
each. The datasets are all in the CoNNL-U format, but are created from different
types of sources. The source types are given as tags such as news, legal, blog,
wiki, etc.

Since the datasets come pre-split, these were concatenated before splitting into
our own sizes. Concatenation happened using the command ``cat *.conllu >
combined.conllu'', since the convention for the datasets is to name the files
something with train, test, and dev, the assumed order of the files is dev,
test, and train. Meaning that if the dev set (eg. the validation set) contains
5000 sentence, only these were used in our datasets, and none of the sentences
from the test or training-sets would be used. No guarantees however were made to
guarantee this ordering, so depending on the naming conventions used in the
specific treebanks this might differ. This shouldn't matter however since there
shouldn't be any difference between the data in the different files.

Some datasets, such as the Norwegian dataset, contains contractions as well as
the individual words. Since the contration is usually unlabelled in the datasets
these were simply removed from the data for ease of parsing. This has the
obvious consequence that the models are not trained on the contractions of words
which are often more commonly used, this however shouldn't affect the
comparisons between word orderings since contractions shouldn't affect these.

The treebanks we selected were all made from news, where some used additional
sources such as non-fiction and spoken. A list of the languages, their
respective treebank selected, and their tags is given below.

* Arabic,   PADT, news

* Hindi,    HDTB, news

* Urdu,     UDTB, news

* Japanese, GSD, blog, news

* Danish,   DDT, fiction, news, non-fiction, spoken

* Norwegian,Bokmaal, blog, news, non-fiction

* Russian,  SynTagRus, fiction, news, non-fiction

The distribution of tokens, labels, etc. is given in appendix.

\subsubsection{Results and analysis}


\pagebreak
