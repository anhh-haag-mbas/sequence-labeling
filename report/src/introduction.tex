
\section{Introduction}

\subsection{Project background}

The project was conducted as a bachelor thesis as part of the Software
Development programme at the IT University of Copenhagen. The project period ran
from February 2nd to May 15th 2019. Originally, the attached supervisor was
Zeljko Agic, whom were involved in deciding the scope and the main problems of
the project. Early in the process however, he was substituted with Leon
Derczynski of the Department of Computer Science at the IT University of
Copenhagen, who supervised the project group untill submission and examination.

The code, report and results of the project are avaible at the projects GitHub
repository at \url{https://github.itu.dk/USS-NLterPrise/sequence-labelling}.


\subsection{Problem description}

The project aim is to perform a large number of empirically based comparisons of
the performance of different machine learning models with well-defined
configurations and implemented in three different open source machine learning
frameworks on a set of languages grouped based on their respective word
ordering. A thorough comparative analysis of the results will be conducted to
possibly identify strengths and weaknesses of the individual frameworks as well
as of the models on different target languages.

A brief summary of the different configurations can be seen in figure (~\ref{}). A
more thorough and motivated description is provided in section (~\ref{}).

The hypothesis of the project is that\ldots


\subsection{Project plan}

The project began in february, and according to the plan, we should have built a
self-flying space rocket on the 21st of december. Also, it should be able to
perform part-of-speech tagging on hindu languages.

\pagebreak
