
\subsection{Configurations}


\subsubsection{Languages}


\subsubsection{Frameworks}

All experiments have been conducted across three popular and open source machine
learning frameworks, two dynamic and one static. Each member of the project
group have had responsibility for the implementation of the models in one of the
frameworks. This section will briefly describe the experience and caveates with
working with each of the frameworks.

This subsection will give a descriptive introduction to each framework, how it
works and the implementation details relevant for this project. For a discussion
and evaluation of each framework, refer to Section~\ref{subsec:frameworks}.


\subsubsection*{DyNet}

This the `magic' framework.


\textbf{The learning proccess}

My general approach to learning dynet was to ease myself into it, following
simple tutorials like an XOR example, and building from there.
With an understanding of how a simple network works I could try to create my own
on a toy problem and see if I really understood how stuff worked.
One of the nice things about Dynet is how close it feels to creating
computational graphs.
Creating parameters and linking them together by applying mathematical
operations on them, made it easy to reason about the behavior of the model.
After the small examples it was simple to learn how to work an LSTM layer into
the model.
LSTM in dynet hides a lot of the complexity associated with the computations,
but the interface was still intuitive to use.
Get the initial state, compute the next state based on input, get the output for
the next layer and save the state for the next part of the sequence.
Working out how to add a CRF layer on top was mostly a matter of understanding
CRF rather than Dynet. 
Figuring out that what we needed for CRF was simply a transformation matrix,
made it obvious that this matrix should just be added asa a parameter similar to
other parts of a model.
From there I could take inspiration from a complete model implemented in dynet
and reuse their code for the CRF.\

\subsubsection*{PyTorch}

This was the dynamic framework that couldn't handle different sized batches.


\subsubsection*{TensorFlow}

Haha, good luck



\subsubsection{Experimental setup}


\pagebreak
